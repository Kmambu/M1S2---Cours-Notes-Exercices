\documentclass[10pt]{article}
\usepackage[utf8]{inputenc}
\usepackage{mathtools}
\usepackage{graphicx}
\usepackage{array}
\usepackage[margin=0.5in]{geometry}
\usepackage{listings}
\usepackage{color}

\definecolor{mygreen}{rgb}{0,0.6,0}
\definecolor{mygray}{rgb}{0.5,0.5,0.5}
\definecolor{mymauve}{rgb}{0.58,0,0.82}

\lstset{ %
  backgroundcolor=\color{white},   % choose the background color; you must add \usepackage{color} or \usepackage{xcolor}; should come as last argument
  basicstyle=\footnotesize,        % the size of the fonts that are used for the code
  breakatwhitespace=false,         % sets if automatic breaks should only happen at whitespace
  breaklines=true,                 % sets automatic line breaking
  captionpos=b,                    % sets the caption-position to bottom
  commentstyle=\color{mygreen},    % comment style
  deletekeywords={...},            % if you want to delete keywords from the given language
  escapeinside={\%*}{*)},          % if you want to add LaTeX within your code
  extendedchars=true,              % lets you use non-ASCII characters; for 8-bits encodings only, does not work with UTF-8
  frame=single,	                   % adds a frame around the code
  keepspaces=true,                 % keeps spaces in text, useful for keeping indentation of code (possibly needs columns=flexible)
  keywordstyle=\color{blue},       % keyword style
  language=Octave,                 % the language of the code
  morekeywords={*,...},            % if you want to add more keywords to the set
  numbers=left,                    % where to put the line-numbers; possible values are (none, left, right)
  numbersep=5pt,                   % how far the line-numbers are from the code
  numberstyle=\tiny\color{mygray}, % the style that is used for the line-numbers
  rulecolor=\color{black},         % if not set, the frame-color may be changed on line-breaks within not-black text (e.g. comments (green here))
  showspaces=false,                % show spaces everywhere adding particular underscores; it overrides 'showstringspaces'
  showstringspaces=false,          % underline spaces within strings only
  showtabs=false,                  % show tabs within strings adding particular underscores
  stepnumber=2,                    % the step between two line-numbers. If it's 1, each line will be numbered
  stringstyle=\color{mymauve},     % string literal style
  tabsize=2,	                   % sets default tabsize to 2 spaces
  title=\lstname                   % show the filename of files included with \lstinputlisting; also try caption instead of title
}
\renewcommand{\arraystretch}{1.5}
\setcounter{secnumdepth}{0}
\author{Kevin Mambu}
\date{\today}
\title{M1 SESI 2017-2018\\Architecture Multi-Processeurs\\TP4 : Partage du Bus
dans les architectures multi-processeurs}

\begin{document}
\maketitle
\section{Spécifications}
Objectifs :
\begin{itemize}
  \item Problèmes de performances posés par le partage du bus.
  \item Goulot d'étranglement lorsque plusieurs processeurs veulent accéder au
  bus car \underline{la bande passante du bus est bornée}.
  \item Temps d'attente du bus proportionnel au nombre de processeurs dans
  l'architecture.
\end{itemize}

\section{B) Architecture Matérielle}
\begin{itemize}
  \item Chaque processeur a un TTY attribué\\
  $\rightarrow$ N TTYs pour un contrôleur de TTY.
  \item $|seg\_tty| = {NPROC}\times{16}$.
  Utilisation d'un framebuffer $({256}\times{256})$ sur 256 niveaux de gris.
  \item Spécifications du framebuffer
  \begin{itemize}
    \item Un tampon de luminence, 64Ko
    \item Un tampon de chrominance, 64Ko, pas utilisé ici
    \item Fréquence de la lecture des tampons : 25 images par seconde
    \item Fréquence d'affichage : $f=\frac{1}{{40}\times{10^-3}}={250}Hz$
  \end{itemize}
\end{itemize}

\subsection{Question B1}
\begin{itemize}
  \item name : nom de l'instance
  \item tgtid : index du signal du framebuffer vis-à-vis du PIBUS
  \item segtab : pointeur vers la table des segments
  \item latency : temps d'accès au framebuffer
  \item width : largeur de l'image (en pixels)
  \item height : hauteur de l'image (en pixels)
  \item subsampling : fréquence de sous'échantillonage de la chrominance
  (ici non-utilisé)
\end{itemize}

\subsection{Question B2}
Le segment associé au Framebuffer doit être aligné et sa taille doit être de
${width}\times{height}$ octets (ici ${256}\times{256}=64$Ko).

\subsection{Dimensionnement des caches}
\begin{itemize}
  \item Caches Write-Through
  \item Correspondance directe
  \item 16 lignes de 8 mots
  \lstinputlisting[firstnumber=29,linerange={29-43},language={C++}]{tp5_top.cpp}
\end{itemize}

\section{C) Compilation de l'application logicielle}
\subsection{Question C1}
Le segment associé au contrôleur du Framebuffer fait partie de l'espace
privilégié. Tout accès mémoire de l'application utilisateur avec le Framebuffer
doit passer par un appel système pour être légal. À cause d'un manque de
permissions de la part de l'application utilisateur, un accès direct avec le
Framebuffer se traduirait par un Data Bus Error.

\subsection{Question C2}
Écrire ligne par ligne limite le nombre de transactions sur le PIBUS à une seule
transaction rafale, une fois le registre intermédiaire plein. Plutôt qu'une
transaction par octet, ce qui est plus coûteux.

\subsection{Question C3}
{\it unsigned int fb\_sync\_write(unsigned int offset, void *buffer,unsigned int length)}\\
\begin{itemize}
  \item offset : le déplacement nécessaire où écrire dans le tampon du
  Framebuffer.\\
  *note : sachant qu'on écrit sur le Framebuffer ligne par ligne, l'offset doit
  être aligné (multiple de 256).
  \item buffer : addresse de base du tampon intermédiaire
  \item : longeur en octets de la donnée à écrire sur le framebuffer.
\end{itemize}

\newpage

\section{D) Caractérisation de l'application logicielle}
\subsection{Question D1}
\lstinputlisting{trace.txt.short}

\subsection{Question D2}
[TODO]

\subsection{Question D3}
[TODO]

\section{E) Exécution sur architecture multi-processeur}
\subsection{Question E1}
\lstinputlisting[language=C,linerange={20-43}]{main.c}

\subsection{Question E2}
\lstinputlisting{reset.s}

\end{document}
