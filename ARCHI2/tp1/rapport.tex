\documentclass[10pt]{article}
\usepackage[utf8]{inputenc}
\usepackage{graphicx}
\usepackage{array}
\usepackage[margin=0.5in]{geometry}
\usepackage{pgf}
\usepackage{tikz}
\usetikzlibrary{arrows,automata}
\renewcommand{\arraystretch}{1.5}
\author{Kevin Mambu}
\date{\today}

\title{M1 SESI 2017-2018\\Architecture Multi-Processeurs\\TP1 : Prototypage Virtuel}

\begin{document}
\maketitle

\section{Énoncé du TP}
Le sujet de ce TP sera le prototypage virtuel, de la simulation extremement fidèle
du système.
L'idée est de reproduire le comportement d'un ordinateur au cycle près et au bit près.

\section{Mode opératoire}
On "construit" une machine complète avec son/ses processeur-s en inteconnectant
entre eux tous les composants matériels d'une vraie machine. \\
Ces composants existent déjà au sein d'une bibliothèque du domaine public appelée
SoClib, utilisant le langage SYSTEM\_C (C++). On aura un processeur de fourni et on
développera, dans un premier temps en MIPS32 puis en C, des programmes à executer
sur notre prototype virtuel.\\

\begin{itemize}
  \item Machine simulante : 1GHz (1 milliard inst/s)
  \item Prototype virtuel : 1Mhz (1 million inst/s)
\end{itemize}

Le prototype est de plus faible cadençage afin de pouvoir rendre l'évolution de la
machine plus macroscopique (on peut faire évoluer le système cycle par cycle) et de
pouvoir inspecter de manière plus fine.\\

\section{Question C1/C2}

\begin{center}

\begin{tabular}{|c|c|}
  \hline
    A & ${SEL}\bullet{ADR\_OK}\bullet{DELAY}\bullet\overline{READ}$ \\ \hline
    B & ${SEL}\bullet\overline{READ}\bullet\overline{DELAY}\bullet{ADR\_OK}$ \\ \hline
    C & ${SEL}\bullet{READ}\bullet\overline{DELAY}\bullet{ADR\_OK}$ \\ \hline
    D & ${SEL}\bullet{ADR\_OK}\bullet{DELAY}\bullet{READ}$ \\ \hline
    E & ${SEL}\bullet\overline{ADR\_OK}$ \\ \hline
    F & $\overline{SEL}$ \\ \hline
    G & \\ \hline
    R & ${SEL}\bullet\overline{ADR\_OK}$ \\ \hline
    S & ${SEL}\bullet{ADR\_OK}$ \\ \hline
    T & $\overline{SEL}\bullet{ADR\_OK}$ \\ \hline
    U & ${GO}$\\ \hline
    V & ${GO}$\\ \hline
    U' & $\overline{GO}$\\ \hline
    V' & $\overline{GO}$\\ \hline
    X & ${SEL}\bullet{ADR\_OK}$ \\ \hline
    Y & ${SEL}\bullet\overline{ADR\_OK}$ \\ \hline
    Z & $\overline{SEL}\bullet{ADR\_OK}$ \\ \hline
\end{tabular}\\[0.8in]

\begin{tabular}{|c|c|c|c|c|}
  \hline
  & {\bf ACK\_EN} & {\bf ACK\_VAL} & {\bf DT\_EN} & {\bf MEM\_CMD} \\ \hline
  {\bf IDLE} & X & 0 & NOP & 0 \\ \hline
  {\bf R\_WAIT} & WAIT & 1 & NOP & 0 \\ \hline
  {\bf R\_OK} & READY & 1 & READ & 1 \\ \hline
  {\bf W\_WAIT} & WAIT & 1 & NOP & 0 \\ \hline
  {\bf W\_OK} & READY & 1 & WRITE & 1 \\ \hline
  {\bf ERROR} & ERROR & 1 & NOP & 0 \\ \hline
\end{tabular}

\end{center}
\section{Question D1/D2}
\begin{tabular}{|c|c|} \hline
  {\bf Label} & {\bf Expr.} \\ \hline
  A &  \\ \hline
  B & ${GNT}$ \\ \hline
  B' & $\overline{GNT}$ \\ \hline
  C &  \\ \hline
  D & ${RDY}$ \\ \hline
  D' & $\overline{RDY}$ \\ \hline
  E & ${RDY}$ \\ \hline
  E' & $\overline{RDY}$ \\ \hline
  F & ${RDY}$ \\ \hline
  F' & $\overline{RDY}$ \\ \hline
  G & ${RDY}$ \\ \hline
  G' & $\overline{RDY}$ \\ \hline
  H & ${GNT}$ \\ \hline
  H' & $\overline{GNT}$ \\ \hline
\end{tabular}
\quad
\begin{tabular}{|c|c|} \hline
  I &  \\ \hline
  J & $\overline{RDY}$ \\ \hline
  K & ${RDY}\bullet{LAST}$ \\ \hline
  L & ${RDY}\bullet\overline{LAST}$ \\ \hline
  M & ${GNT}$ \\ \hline
  M' & $\overline{GNT}$ \\ \hline
  N &  \\ \hline
  O & $\overline{RDY}$ \\ \hline
  P & ${RDY}\bullet{NUL}$ \\ \hline
  Q & ${RDY}\bullet\overline{NUL}$ \\ \hline
  R & ${GNT}$ \\ \hline
  R' & $\overline{GNT}$ \\ \hline
  S &  \\ \hline
  T & ${RDY}$ \\ \hline
  T' & $\overline{RDY}$ \\ \hline
\end{tabular}\\[0.5in]
\begin{tabular}{|c|c|c|c|c|c|c|}
  \hline
  & REQ & CMD\_EN & ADR\_VALUE & READ\_VALUE & LOCK\_VALUE & DT\_EN \\ \hline
  {\bf INIT} & 0 & 0 & X & X & X & 0 \\ \hline
  {\bf RAM\_REQ} & 1 & 0 & X & X & X & 0 \\ \hline
  {\bf RAM\_A0} & 0 & 1 & RAM\_BASE & 1 & 1 & 0 \\ \hline
  {\bf RAM\_A1\_D0} & 0 & 1 & $RAM\_BASE + 4$ & 1 & 1 & 0 \\ \hline
  {\bf RAM\_A2\_D1} & 0 & 1 & $RAM\_BASE + 8$ & 1 & 1 & 0 \\ \hline
  {\bf RAM\_A3\_D2} & 1 & 1 & $RAM\_BASE + 12$ & 1 & 0 & 0 \\ \hline
  {\bf RAM\_D3} & 0 & 0 & X & 1 & 0 & 0 \\ \hline
  {\bf W\_REQ} & 1 & 0 & X & X & X & 0 \\ \hline
  {\bf W\_AD} & 0 & 1 & $SEG\_TTY\_BASE$ & 0 & 0 & 0 \\ \hline
  {\bf W\_DT} & 0 & 1 & $SEG\_TTY\_BASE$ & 0 & 0 & 1 \\ \hline
  {\bf STS\_REQ} & 1 & 0 & X & X & 0 & 0 \\ \hline
  {\bf STS\_AD} & 0 & 1 & $SEG\_TTY + 4$ & X & X & 0 \\ \hline
  {\bf STS\_DT} & 0 & 1 & X & ? & X & ? \\ \hline
  {\bf BUF\_REQ} & 1 & 0 & X & X & X & 0 \\ \hline
  {\bf BUF\_AD} & 0 & 1 & X & X & X & 1 \\ \hline
  {\bf BUF\_DT} & 0 & 1 & X & 1 & X & 1 \\ \hline
\end{tabular}

\newpage

\section{Question E1/E2}
\begin{tabular}{|c|c|}
  \hline
  {\bf Label} & {\bf Expr.} \\ \hline
  X' & $\overline{REQ}$ \\ \hline
  X & ${REQ}$ \\ \hline
  Y' & $\overline{LOCK}$ \\ \hline
  Y & ${LOCK}$ \\ \hline
  Z' & $\overline{({ACK}\oplus{WAIT})}+{LOCK}$ \\ \hline
  Z & $\overline{LOCK}\bullet({ACK}\oplus{WAIT})$ \\ \hline
  K & $\overline{REQ}$ \\ \hline
  L & ${REQ}$ \\ \hline
\end{tabular}
\quad
\begin{tabular}{|c|c|c|c|}
  \hline
  & GNT & SEL0 & SEL1 \\ \hline
  IDLE & 0 & 0 & 0 \\ \hline
  AD & 1 & $(A>>msb\_shift) == 0$ & $(A>>msb\_shift) == 1$ \\ \hline
  DTAD & 1 & $(A>>msb\_shift) == 0$ & $(A>>msb\_shift) == 1$ \\ \hline
  DT & 0 & 0 & 0 \\ \hline
\end{tabular}

\section{Question E3}
Lorsque l'état courant du BCU est à l'état IDLE, si après envoi de la dernière
réponse il restait une requête suivante dans l'ordre de priorité, il faut la
traiter au cycle suivant. Le BCU ne retourne dans létat IDLE que lorsqu'il est
en attente d'une requête maître.
\end{document}
