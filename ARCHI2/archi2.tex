\documentclass[10pt]{article}
\usepackage{hyperref}
\usepackage[utf8]{inputenc}
\usepackage{mathtools}
\usepackage{multicol}
\usepackage{amsmath}
\usepackage{graphicx}
\usepackage{array}
\usepackage[margin=0.5in]{geometry}
\usepackage{listings}
\usepackage{color}

\definecolor{mygreen}{rgb}{0,0.6,0}
\definecolor{mygray}{rgb}{0.5,0.5,0.5}
\definecolor{mymauve}{rgb}{0.58,0,0.82}

\lstset{ %
  backgroundcolor=\color{white},   % choose the background color; you must add \usepackage{color} or \usepackage{xcolor}; should come as last argument
  basicstyle=\footnotesize,        % the size of the fonts that are used for the code
  breakatwhitespace=false,         % sets if automatic breaks should only happen at whitespace
  breaklines=true,                 % sets automatic line breaking
  captionpos=b,                    % sets the caption-position to bottom
  commentstyle=\color{mygreen},    % comment style
  deletekeywords={...},            % if you want to delete keywords from the given language
  escapeinside={\%*}{*)},          % if you want to add LaTeX within your code
  extendedchars=true,              % lets you use non-ASCII characters; for 8-bits encodings only, does not work with UTF-8
  frame=single,	                   % adds a frame around the code
  keepspaces=true,                 % keeps spaces in text, useful for keeping indentation of code (possibly needs columns=flexible)
  keywordstyle=\color{blue},       % keyword style
  language=Octave,                 % the language of the code
  morekeywords={*,...},            % if you want to add more keywords to the set
  numbers=left,                    % where to put the line-numbers; possible values are (none, left, right)
  numbersep=5pt,                   % how far the line-numbers are from the code
  numberstyle=\tiny\color{mygray}, % the style that is used for the line-numbers
  rulecolor=\color{black},         % if not set, the frame-color may be changed on line-breaks within not-black text (e.g. comments (green here))
  showspaces=false,                % show spaces everywhere adding particular underscores; it overrides 'showstringspaces'
  showstringspaces=false,          % underline spaces within strings only
  showtabs=false,                  % show tabs within strings adding particular underscores
  stepnumber=2,                    % the step between two line-numbers. If it's 1, each line will be numbered
  stringstyle=\color{mymauve},     % string literal style
  tabsize=2,	                   % sets default tabsize to 2 spaces
  title=\lstname                   % show the filename of files included with \lstinputlisting; also try caption instead of title
}
\renewcommand{\arraystretch}{1.5}
\setcounter{secnumdepth}{0}
\author{Alain Greiner\\{\it rédigé par Kevin Mambu}}
\date{\today}

\title{M1 SESI 2017-2018\\Architecture Multi-Processeurs\\Notes de Cours}

\begin{document}
\maketitle

\newpage
\tableofcontents

\newpage

\section{Introduction}
Enseignants :
\begin{itemize}
  \item Franck Wajsbürt
  \item Lionel Lacasagne
\end{itemize}
MdP Escanglon 226 : [modifié]\\
A la base, le module s'intéressait à aux bs de communications de l'Ordinateur (
bus PCI, bridges, etc.).\\
Les bus systèmes sont de moins en moins exposés $\Rightarrow$ switch vers l'étude
de réseaux sur puce, systèmes embarqués sans fil, IoT.\\

Pré-requis l'élaboration d'un système :
\begin{itemize}
  \item Définition des besoins du dispositif, pour élaborer des solutions
  \item Maîtrise de conception matérielle
  \item Maîtrise de conception logicielle
  \item Maîtrise de conception système
\end{itemize}

Points abordés :
\begin{itemize}
  \item Drivers Linux
  \item Communication proccess
  \item Communication sans fil
  \item Plate-forme Arduino
  \item Programmation basse consommation
\end{itemize}

\section{Présentation du projet semestriel}

On aura des capteurs sans fil relié sur un module (Arduino) communicant par un
relais sans fil (un Raspberry Pi) et qui aura à charge de retransmettre des
informations.\\
Les capteurs sont respectivement à technologie Bluetooth \& nRF24L01. Le dernier
dispose de certains avantages : il est moins onéreux en qualité de composants et
décrit un protocole de transfert de paquets plus simplifié. La technologie
Bluetooth est plus automatisé.
Certains modules integrent directement un SoC, comme un $\mu$-controlleur, d'autres
doivent passer par un module annexe.\\
On s'est abstenu d'utiliser la technologie Wi-Fi pour certaines raisons : il s'agit
d'une technologie onéreuse d'un point de vue énergétique (100mA de consommé pour
une connection).\\
On mettra sur la RPi une Gateway Pi $\rightarrow$ Modules et on va y installer un
serveur afin de communiquer avec des clients sur le web via connection ethernet.
Il y aura en conséquence une partie gestion des clients et des bases de données.\\
Pednant la programmation de ces modules sur RPi, il faudra faire des choix
logiciels dans la programmation. $\Rightarrow$ programmation consciente de la
consommation énergétique.\\
Idée de Franck $\rightarrow$ utiliser une capacité en déchargement pour réduire
la consommation.\\
Même avec le système opérationnel matériellement, il faudra appliquer des filtres
et un nettoyage du bruit, pour bonne emission de l'information. Néanmoins, ces
opérations de filtrage sont coûteuses énergétiquement, il faudra donc faire
compromis. Une fois ces filtres fait sur RPi, il faudra les restituer sur le
module Arduino.

\newpage
\section{Introduction}
Enseignants :
\begin{itemize}
  \item Franck Wajsbürt
  \item Lionel Lacasagne
\end{itemize}
MdP Escanglon 226 : [modifié]\\
A la base, le module s'intéressait à aux bs de communications de l'Ordinateur (
bus PCI, bridges, etc.).\\
Les bus systèmes sont de moins en moins exposés $\Rightarrow$ switch vers l'étude
de réseaux sur puce, systèmes embarqués sans fil, IoT.\\

Pré-requis l'élaboration d'un système :
\begin{itemize}
  \item Définition des besoins du dispositif, pour élaborer des solutions
  \item Maîtrise de conception matérielle
  \item Maîtrise de conception logicielle
  \item Maîtrise de conception système
\end{itemize}

Points abordés :
\begin{itemize}
  \item Drivers Linux
  \item Communication proccess
  \item Communication sans fil
  \item Plate-forme Arduino
  \item Programmation basse consommation
\end{itemize}

\section{Présentation du projet semestriel}

On aura des capteurs sans fil relié sur un module (Arduino) communicant par un
relais sans fil (un Raspberry Pi) et qui aura à charge de retransmettre des
informations.\\
Les capteurs sont respectivement à technologie Bluetooth \& nRF24L01. Le dernier
dispose de certains avantages : il est moins onéreux en qualité de composants et
décrit un protocole de transfert de paquets plus simplifié. La technologie
Bluetooth est plus automatisé.
Certains modules integrent directement un SoC, comme un $\mu$-controlleur, d'autres
doivent passer par un module annexe.\\
On s'est abstenu d'utiliser la technologie Wi-Fi pour certaines raisons : il s'agit
d'une technologie onéreuse d'un point de vue énergétique (100mA de consommé pour
une connection).\\
On mettra sur la RPi une Gateway Pi $\rightarrow$ Modules et on va y installer un
serveur afin de communiquer avec des clients sur le web via connection ethernet.
Il y aura en conséquence une partie gestion des clients et des bases de données.\\
Pednant la programmation de ces modules sur RPi, il faudra faire des choix
logiciels dans la programmation. $\Rightarrow$ programmation consciente de la
consommation énergétique.\\
Idée de Franck $\rightarrow$ utiliser une capacité en déchargement pour réduire
la consommation.\\
Même avec le système opérationnel matériellement, il faudra appliquer des filtres
et un nettoyage du bruit, pour bonne emission de l'information. Néanmoins, ces
opérations de filtrage sont coûteuses énergétiquement, il faudra donc faire
compromis. Une fois ces filtres fait sur RPi, il faudra les restituer sur le
module Arduino.

\newpage
\section{Introduction}
Enseignants :
\begin{itemize}
  \item Franck Wajsbürt
  \item Lionel Lacasagne
\end{itemize}
MdP Escanglon 226 : [modifié]\\
A la base, le module s'intéressait à aux bs de communications de l'Ordinateur (
bus PCI, bridges, etc.).\\
Les bus systèmes sont de moins en moins exposés $\Rightarrow$ switch vers l'étude
de réseaux sur puce, systèmes embarqués sans fil, IoT.\\

Pré-requis l'élaboration d'un système :
\begin{itemize}
  \item Définition des besoins du dispositif, pour élaborer des solutions
  \item Maîtrise de conception matérielle
  \item Maîtrise de conception logicielle
  \item Maîtrise de conception système
\end{itemize}

Points abordés :
\begin{itemize}
  \item Drivers Linux
  \item Communication proccess
  \item Communication sans fil
  \item Plate-forme Arduino
  \item Programmation basse consommation
\end{itemize}

\section{Présentation du projet semestriel}

On aura des capteurs sans fil relié sur un module (Arduino) communicant par un
relais sans fil (un Raspberry Pi) et qui aura à charge de retransmettre des
informations.\\
Les capteurs sont respectivement à technologie Bluetooth \& nRF24L01. Le dernier
dispose de certains avantages : il est moins onéreux en qualité de composants et
décrit un protocole de transfert de paquets plus simplifié. La technologie
Bluetooth est plus automatisé.
Certains modules integrent directement un SoC, comme un $\mu$-controlleur, d'autres
doivent passer par un module annexe.\\
On s'est abstenu d'utiliser la technologie Wi-Fi pour certaines raisons : il s'agit
d'une technologie onéreuse d'un point de vue énergétique (100mA de consommé pour
une connection).\\
On mettra sur la RPi une Gateway Pi $\rightarrow$ Modules et on va y installer un
serveur afin de communiquer avec des clients sur le web via connection ethernet.
Il y aura en conséquence une partie gestion des clients et des bases de données.\\
Pednant la programmation de ces modules sur RPi, il faudra faire des choix
logiciels dans la programmation. $\Rightarrow$ programmation consciente de la
consommation énergétique.\\
Idée de Franck $\rightarrow$ utiliser une capacité en déchargement pour réduire
la consommation.\\
Même avec le système opérationnel matériellement, il faudra appliquer des filtres
et un nettoyage du bruit, pour bonne emission de l'information. Néanmoins, ces
opérations de filtrage sont coûteuses énergétiquement, il faudra donc faire
compromis. Une fois ces filtres fait sur RPi, il faudra les restituer sur le
module Arduino.

\end{document}
