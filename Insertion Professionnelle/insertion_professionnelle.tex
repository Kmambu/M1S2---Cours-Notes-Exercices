\documentclass[10pt]{article}
\usepackage[utf8]{inputenc}
\usepackage{graphicx}
\author{Kevin Mambu}
\date{\today}

\title{Insertion et Orientation Professionnelle\\Introduction}

\begin{document}
\maketitle

\tableofcontents

\newpage

\section{Introduction}
{\it Présentée par Colette LUCAS}\\
Responsable de l'UE "Orientation et Insertion Professionnelle". Il faut choisir
aux moins trois conférences, pour valider l'UE. Les conférences sont sur le site
du Master, les dates déjà fixées pour les six premières.\\
Objectif : $\Rightarrow$ faciliteer notre compréhension du monde professionnelle
et l'insertion professionnelle.\\
WARN : il faut dire "Faculté de Sciences et d'Ingénierie blablabla..."\\
\section{Les atouts pour une insertion professionnelle}
\subsection{Les atouts de l'etablissement}
\begin{itemize}
  \item 35000 étudiants dont 20\% d'étrangers
  \item 23300 étudiants dans le domaine scientifique
  \item 97 laboratoires associés au CRNS, l'INRA, l'INSERM
  \item des sites tels que Saint-Cyr, l'institut d'Astrophysique
  \item 2200 Bac+5 délivrés par an
\end{itemize}

Un problème réccurrent $\Rightarrow$ comparaison des étudiants de faculté avec
les élèves d'école d'ingénierie.

\subsection{Les atouts de l'étudiant}
Les recruteurs recherchent :
\begin{itemize}
  \item Un potentiel d'innovation
  \item De l'adaptabilité
  \item De l'autonomie
\end{itemize}

Un étudiant de Sorbonne Université, lui, répond à ces attentes grâce à :
\begin{itemize}
  \item D'enseignents ET de chercheurs $\Rightarrow$ MÀJ permanente de l'enseignement
scientifique par les enseigneurs.
  \item D'un enseignement fondamental et applicatif, avec un mode opératoire
rigoureux, aussi bien pour la conception que l'expérimentation.
  \item De compléments en Insertion Professionnelle (ex: Atrium des Métiers,
Conférence en Insetion Professionnelle, Conférences de professionnels de l'industrie)
  \item D'un sens de l'autonomie, grâce à ses confrontations à des choix, comme
par exemple les choix d'intégration, d'insertion, de spécialisation
\end{itemize}

\subsection{Les atouts du doctorat}
\begin{itemize}
  \item Production de connaisances nouvelles
  \item Persévérence et aboutissements sur un "dossier"
  \item Gestion de la complexité
  \item Confrontation de l'imprévu
  \item Exploration de voies alternatives
  \item État d'esprit ouvert au doute $\Rightarrow$ savoir se remettre en question
  \item Travail en collaboration
  \item Appartenance à une communauteé interntionale
\end{itemize}
\section{L'insertion Professionnelle}

Nb : chercher un stage $\Rightarrow$ discuter avec les chargés d'Insertion
Professionnelle. Ces chargés peuvent être dédicacés comme des enseignants-
chercheurs, le Décannat, les responsables d'UFR. Un groupe d'environ 25
personnes.

\subsection{Interaction avec le domaine professionnel par la faculté}
\begin{itemize}
  \item DFC (Doctorants)
  \item Conseil de perfectionnement : une vingtaine de personnes chargées de
mettre au cours du jour de l'industrie les enseignements Ex : les UE d'ingénierie.
  \item Gouvernance, Vice-Doyen F\&IP, Chargés de Missions \& Relations
\end{itemize}

\subsubsection{Connaître ses formations}
\begin{itemize}
  \item Auprès des acteurs industriels, il faut savoir se valoriser, décrire ses
formations, afficher les spécificités et les atouts de son établissement.
  \item Auprès de ses pairs, il faut savoir clarifier les informations sur sa
formation
\end{itemize}

\subsubsection{Les résultats de votre établissement}

Connaître son établissement par les chiffres, exemples :
\begin{itemize}
  \item Classement de Shanghai (FSI Sorbonne : 28 clasements/52, dans le top 50)
\end{itemize}
{\it Mal nommer les choses, c'est ajouter à la misère du Monde\\- Albert Camus}

\subsection{Mieux connaître le monde professionnel}
\begin{itemize}
  \item Aller au contact des acteurs industriels, ne pas hésiter à les questionner
  \item {\bf Identifier, Transmettre les codes (linguistiques, comportementaux,
vestimentaires)}
\end{itemize}

\subsubsection{Soft-Skills}
Ensemble des étiquettes socio-professionnelles à appliquer pour un bon contact
relationnel
\subsection{Mieux SE connaître}
\begin{itemize}
  \item savoir
  \item pouvoir
  \item vouloir
  \item devoir
  \item aimer... wtf
\end{itemize}
{\it Compétence + Préférence = Performance}. Analyser ses succès et savoir quelles
préférences en sont à l'origine.\\
{\it Une qualité quelque part peut-être un défaut ailleurs}
\end{document}
